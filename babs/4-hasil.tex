
\newpage
\chapter{Hasil}
 
Hasil berfungsi untuk melaporkan hasil pelaksanaan metode/teknik penelitian dan menyajikan data yang mendukung hasil tersebut. Penyajian data dan penjelasannya dilakukan secara terurut dan logis menggunakan teks dan ilustrasi lainnya (misalnya, tabel dan gambar). Urutan penjelasan dapat dilakukan secara kronologis berdasarkan urutan pelaksanaan metode atau berdasarkan tingkat kepentingan substansinya, dari yang lebih penting sampai ke yang proritasnya lebih rendah. 

\section{Subbab Dua Satu}

Sebelum menuliskan hasil ke dalam laporan, perlu dicermati dan ditentukan mana hasil yang relevan dan dapat digunakan untuk menjawab pertanyaan atau masalah penelitian.  Hasil inilah yang perlu dimasukkan terlepas dari apakah hasil ini positif (misalnya, mendukung kebenaran hipotesis) atau negatif (misalnya, menolak hipotesis). Selanjutnya, perlu diperhatikan bagaimana menyajikannya dengan cara terbaik, apakah dengan teks, tabel atau gambar. Tabel dan gambar (foto, gambar, grafik, diagram) sering digunakan untuk mempresentasikan data yang detil dan kaya, sementara teks digunakan untuk menarasikan temuan yang lebih umum dan menjelaskan bagian-bagian tertentu yang menjadi fokus dalam tabel dan gambar. 

\begin{table}
  \centering
  \renewcommand{\arraystretch}{1.2}
  \caption{Tabel Judul Pendek di BAB 4}
  \begin{tabular}{clc}
    \hline
    No & Keanggotaan IMT & Rentang Nilai \\
    \hline
    1 & Sangat Kurus & 0.0 - 19.0 \\
    2 & Kurus & 15.0 - 20.0 \\
    \hline
    \multicolumn{3}{l}{\footnotesize{Sumber: Diadaptasi dari \cite{anggariawan:2014}}} \\
  \end{tabular}
  \label{tab:dibab4}
\end{table}


\section{Subbab Dua Dua}

Hasil dan pembahasan dapat diletakkan dengan kemungkinan berikut:
\begin{enumerate}
  \item	Dipisahkan secara fisik ke dalam bab-bab yang berbeda
  \item Dipisahkan secara fisik ke dalam dua atau lebih paragraf atau subbab yang berbeda tetapi dalam bab yang sama
  \item Dileburkan menjadi satu dalam paragraf, dijelaskan secara naratif-deskriptif, terdistribusi ke satu atau lebih bab yang ada
\end{enumerate}

\subsection{Subbab Empat Dua Satu}

Cara pertama atau kedua membantu pembaca yang ingin memisahkan observasi dan terjemahan dari observasi tersebut sehingga mereka dapat menilai kualitas dari masing-masing proses dengan lebih mudah. Kadang-kadang cara kedua lebih banyak dipilih daripada cara pertama jika data yang harus dipresentasikan yang cukup banyak dan laporan penelitian cukup panjang agar pembaca tidak perlu menunggu presentasi dari seluruh data selesai baru dapat membaca penerjemahannya. Cara pertama dan kedua ini banyak digunakan untuk penelitian yang bersifat kuantitatif, baik itu deskriptif, analitik, maupun implementatif.    

\subsection{Subbab Empat Dua Dua}

Cara ketiga biasanya digunakan jika data, analisis, dan penafsirannya sulit dipisahkan. Pemisahannya terkadang justru membuat laporan penelitian sulit dibaca. Hal ini dapat berlaku pada tipe penelitian yang bersifat kualitatif, baik itu deskriptif ataupun analitik/eksplanatori. 
Pada dasarnya peletakan dan jumlah bab untuk hasil dan pembahasan sebaiknya disesuaikan karakter penelitian masing-masing. Judul bab pun tidak harus secara eksplisit "Hasil" dan "Pembahasan" tetapi dapat digantikan dengan nama yang lebih deskpritif dan tematik. 

\section{Subbab Empat Tiga}

Contoh struktur tesis untuk implementatif pembangunan dan nonimplementatif dapat dilihat pada kedua subbab berikut. 

\subsection{Contoh Struktur Penelitian Implementatif Pengembangan}

Berikut ini adalah contoh bab-bab yang terdapat pada penelitian implementatif pengembangan sistem perangkat lunak. 
\begin{displayquote}
  Bab 1 Pendahuluan \\
  Bab 2 Landasan kepustakaan \\
  Bab 3 Metodologi penelitian \\
  Bab 4 Rekayasa persyaratan/kebutuhan \\ 
  Bab 5 Perancangan dan implementasi \\
  Bab 6 Pengujian \\
  Bab 7 Penutup \\
\end{displayquote}

Bab 1 sampai Bab 3 memuat informasi yang sesuai dengan panduan sebelumnya. Isi dari bab-bab berikutnya disesuaikan dengan syarat kecukupan tesis untuk tipe implementatif berdasarkan keminatan masing-masing, seperti yang terdapat pada panduan kecukupan tesis dan aturan khusus dari keminatan masing-masing. Di bawah ini adalah sebuah contoh saja: 

\begin{displayquote}
  Bab 4 Persyaratan:
  \begin{itemize}
    \item Pernyataan masalah yang lebih elaboratif/mendetail daripada yang di Pendahuluan.
    \item Identifikasi pemangku kepentingan (stakeholders) dan aktor (actors) sistem.
    \item Daftar terstruktur persyaratan/kebutuhan perangkat lunak, secara fungsional, data, dan nonfungsional
    \item Use cases, use case diagrams, use case specifications, dan sebagainya. 
  \end{itemize} 
  Bab 5 Perancangan dan implementasi:
  \begin{itemize}
    \item Rancangan arsitektur: detesis struktur dan setiap komponen utama
    \item Representasi data dalam model data dan basis data
    \item Detil implementasi dari fungsi-fungsi utama yang menjadi fokus
  \end{itemize}
  Bab 6 Pengujian dan evaluasi
  \begin{itemize}
    \item Strategi, rencana, kasus, dan data pengujian
    \item Ringkasan hasil pengujian perangkat lunak, termasuk data dan analisisnya (detilnya di Lampiran)
    \item Evaluasi hasil proyek secara keseluruhan, misalkan 
  \end{itemize}
  Bab 7 Penutup
  \begin{itemize}
    \item Ringkasan dari capaian proyek
    \item Saran pengembangan lebih lanjut
  \end{itemize}
\end{displayquote}

Pada contoh struktur ini "hasil" tersebar di beberapa bab mulai Bab 4 Persyaratan sampai Bab 6, sedangkan "pembahasan" secara keseluruhan terhadap masalah penelitian terdapat di Bab 6. Yang dimaksud dengan pengujian dalam Bab 6 terfokus pada pengujian persyaratan perangkat lunak, sedangkan evaluasi berfungsi sebagai "pembahasan" secara keseluruhan, yaitu menentukan apakah "hasil" sudah menjawab masalah penelitian yang dirumuskan pada Bab 1. 

Sebagai catatan, Bab 3 Metodologi Penelitian umumnya menjelaskan model proses perangkat lunak yang digunakan. Jika strategi untuk setiap aktivitasnya (analisis persyaratan, perancangan, dan seterusnya) sudah dijelaskan di Bab 3 ini juga, maka bab-bab lainnya yang berhubungan dengan aktivitas-aktivitas ini masing-masing langsung dapat menjelaskan hasil pelaksanaan metodenya. 

\subsection{Contoh Struktur Penelitian Nonimplementatif}

Berikut ini adalah contoh bab-bab yang terdapat pada penelitian nonimplementatif. 

\begin{displayquote}
  Bab 1 Pendahuluan \\
  Bab 2 Landasan kepustakaan \\
  Bab 3 Metodologi Penelitian \\
  Bab 4 Hasil \\
  Bab 5 Pembahasan \\
  Bab 6 Penutup
\end{displayquote}

Isi dari setiap bab dapat menyesuaikan dengan panduan yang telah dijelaskan sebelumnya. Jika diperlukan, Bab 4 dapat digabungkan dengan Bab 5, menjadi Hasil dan Pembahasan. 

Struktur dasar ini cukup universal sehingga dapat digunakan juga untuk tipe-tipe penelitian lainnya, khususnya jika belum ada struktur lain yang lebih tematik dan cocok untuk penelitian yang bersangkutan. Untuk lebih tepatnya, struktur penulisan menyesuaikan dengan studi keminatan serta saran dari dosen pembimbing masing-masing.

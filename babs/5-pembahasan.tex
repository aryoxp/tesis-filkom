\newpage
\chapter{Pembahasan}

Pembahasan berfungsi untuk menerjemahkan makna dari hasil yang diperoleh untuk menjawab pertanyaan atau masalah penelitian. Fungsi lainnya adalah untuk menjelaskan pemahaman baru yang didapatkan dari hasil penelitian, yang diharapkan berguna dalam pengembangan keilmuan. Dalam penelitian tingkat lanjut, fungsi pembahasan yang kedua ini sangat penting karena dapat menunjukkan kontribusi penulis terhadap pengembangan keilmuan. Akan tetapi, dalam penelitian tingkat tesis, fungsi yang kedua ini dapat diterapkan secara terbatas karena pendidikan S1 tidak dituntut untuk pengembangan keilmuan secara substansial, tetapi cukup terhadap pemahaman personal dalam implementasi konsep atau teori. 

\section{Subbab Lima Satu}

Dalam menjawab masalah penelitian, penulis diminta untuk melakukan evaluasi kritis terhadap hasil yang diperoleh. Tergantung dari fokus penelitian, beberapa contoh pertanyaan kritis yang dapat dijawab adalah:
\begin{itemize}
  \item Seberapa jauh tujuan penelitian telah tercapai?
  \item Apakah aplikasi atau sistem yang dibangun sesuai dengan tujuannya?
  \item Apakah metode atau praktik perancangan dan implementasi yang baik telah dijalankan?
  \item Apakah teknologi implementasi yang tepat telah dipilih? Dan sebagainya. 
\end{itemize} 

\section{Subbab Lima Dua}

Dalam menjelaskan pemahaman baru yang didapatkan, penulis dapat mengubungkan hasil penelitian dengan pengetahuan teoritik atau penelitian sebelumnya yang telah dibahas. Kaitan antara hasil penelitian dan pengetahuan teoritik misalnya berupa:
\begin{itemize}
  \item pendapat tentang metode yang digunakan dari pustaka, apakah dapat digunakan dengan baik secara langsung, dengan penyesuaian, atau dengan batasan tertentu;
  \item konfirmasi tentang batasan dari metodologi yang digunakan sehingga dapat berpengaruh pada hasil;
  \item penjelasan tentang informasi penting pada penelitian lainnya yang membantu penulis untuk menerjemahkan data penelitian penulis;
  \item penjelasan tentang kemungkinan hasil dari penelitian lainnya yang dapat dikombinasikan dengan penelitian penulis untuk memberikan pengetahuan baru; dan sebagainya. 
\end{itemize}

\section{Subbab Lima Tiga}
Penulis dapat merefleksikan apa yang telah dipelajari selama melakukan penelitian, tetapi harus tetap terfokus dengan masalah penelitian ini dan tidak melebar ke masalah lainnya. Hal-hal yang berada di luar fokus peneltian tetapi penting dan menarik untuk diteliti dapat disarankan sebagai bahan penelitian berikutnya. Hal ini dapat dipertegas di bab Kesimpulan/ Penutup. 

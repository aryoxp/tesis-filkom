\newpage
\centering{
  \bfseries\LARGE{
    ABSTRAK \\
  }
}
\addcontentsline{toc}{section}{\MakeUppercase{Abstrak}}
\vspace{10mm}
\normalsize
\justifying

% \noindent
% \textbf{\pnama, \ptitle}
% \textbf{
%   \noindent
%   \\Pembimbing: \ppembimbingsatu
%   \ifthenelse{\equal{\ppembimbingdua}{}}
%   {} {\phantom{ }dan \ppembimbingdua}}

% \vspace{5mm}

\indent
Bagian ini diisi dengan abstrak dalam Bahasa Indonesia. Abstrak adalah uraian singkat (umumnya 200-300 kata) yang merupakan intisari dari sebuah tesis. Abstrak membantu pembaca untuk mendapatkan gambaran secara cepat dan akurat tentang isi dari sebuah tesis. Melalui abstrak, pembaca juga dapat menentukan apakah akan membaca tesis lebih lanjut. Oleh karena itu, abstrak sebaiknya memberikan gambaran yang padat tetapi tetap jelas dan akurat tentang (1) apa dan mengapa penelitian dikerjakan: sedikit latar belakang, pertanyaan atau masalah penelitan, dan/atau tujuan penelitian; (2) bagaimana penelitian dikerjakan: rancangan penelitian dan metodologi/metode dasar yang digunakan dalam penelitian; (3) hasil penting yang diperoleh: temuan utama, karakteristik artefak, atau hasil evaluasi artefak yang dibangun; (4) hasil pembahasan dan kesimpulan: hasil dari analisis dan pembahasan temuan atau evaluasi artefak yang dibangun, yang dikaitkan dengan pertanyaan/tujuan penelitian.

Yang harus dihindari dalam sebuah abstrak diantaranya (1) penjelasan latar belakang yang terlalu panjang; (2) sitasi ke pustaka lainnya; (3) kalimat yang tidak lengkap; (3) singkatan, jargon, atau istilah yang membingungkan pembaca, kecuali telah dijelaskan dengan baik; (4) gambar atau tabel; (5) angka-angka yang terlalu banyak.

Di akhir abstrak ditampilkan beberapa kata kunci (normalnya 5-7) untuk membantu pembaca memposisikan isi tesis dengan area studi dan masalah penelitian. Kata kunci, beserta judul, nama penulis, dan abstrak biasanya dimasukkan dalam basis data perpustakaan. Kata kunci juga dapat diindeks dalam basis data sehingga dapat digunakan untuk proses pencarian tulisan ilmiah yang relevan. Oleh karena itu pemilihan kata kunci yang sesuai dengan area penelitian dan masalah penelitian cukup penting. Pemilihan kata kunci juga bisa didapatkan dari referensi yang dirujuk. 

\vspace{5mm}
\noindent
Kata kunci: abstrak, tesis, intisari, kata kunci, artefak
